\documentclass[12pt,spanish,letterpaper]{article}
\usepackage[T1]{fontenc}
\usepackage[latin1]{inputenc}
\usepackage{palatino}
\usepackage{ae}
\usepackage[spanish]{babel}
\usepackage{xparse}
\usepackage{latexsym}
\usepackage{verbatim}
\usepackage{moreverb}
\usepackage{color}
\usepackage{babel}
\usepackage{setspace}%--->\doublespace ---> para interlineado 1,5
%\usepackage{newcent}
\usepackage[dvips]{graphicx}
\usepackage{epsfig}
\usepackage{epsf}
\usepackage{multicol}
\usepackage{indentfirst}
%\setlength{\parindent}{8pt}  %---> para el tama�o del margen de la primera palabra del parrafo
\usepackage{amsthm}
\usepackage{anysize}
%\usepackage{float}%%%%=== Se usa ahora [H] en vez de [h!], aunque se pede seguir usando este ultimo.
%\usepackage{pslatex}%---> Tipo de letra Times New Roman

\marginsize{2cm}{2cm}{2cm}{2cm}
\renewcommand{\baselinestretch}{1.2}
\setlength{\parindent}{.5cm}

%%%%%%%%%%%%%%%%%%%%%%%%%%%%%%%%%%%%%%%%%%%%%%%%%%%%%%%%%%%%%%%%%%%
%%%%%%%%%%%---INICIO DE ENCABEZADO FANCY---%%%%%%%%%%%%%%%%%%%%%%%%
%\usepackage{fancybox}
%\usepackage{helvet}
%\newcommand{\helv}{\fontfamily{phv}\fontsize{9}{11}\selectfont}
%\usepackage{fancyhdr}
%\pagestyle{fancy}
%% Definir las marcas: cap�tulo. secci�n -------
%%\renewcommand{\chaptermark}[1]{\markboth{#1}{}}
%\renewcommand{\sectionmark}[1]{\markright{Osmani Tito Corrioso: Sistema de Torneo GPA}}
%\renewcommand{\subsectionmark}[1]{\markright{Osmani Tito Corrioso: Sistema de Torneo GPA}}
%\renewcommand{\footnotemark}[1]{\markright{EduSol}}
%\fancyhf{} % borra cabecera y pie actuales
%% El n�mero de p�gina
%\fancyhead[LE,RO]{\helv\thepage} %L=Left, R=right, O=Odd ( impar) , E=Even p�gs pares
%% "Marcas" a la derecha e izquierda del encabezado
%\fancyhead[LO]{\helv\rightmark}
%\fancyhead[RE]{\helv\leftmark}
%\fancyfoot[CE,CO]{\helv EduSol}
%%\fancyfoot[LO]{\helv\rightmark}
%%\fancyfoot[RE]{\helv\leftmark}
%\renewcommand{\headrulewidth}{1pt} % Sin raya. Con raya?: cambiar {0} por {0.5pt}
%\renewcommand{\footrulewidth}{1pt}
%\addtolength{\headheight}{0.5pt} % espacio para la raya
%%\fancypagestyle{plain}{ %
%%\fancyhead{} % elimina cabeceras y raya en p�ginas "plain"
%%\renewcommand{\headrulewidth}{0pt}}
%%%%%%%%%%%%%%%%%%%%%%%%%---END---%%%%%%%%%%%%%%%%%%%%%%%%%%%%%%%%%
%%%%%%%%%%%%%%%%%%%%%%%%%%%%%%%%%%%%%%%%%%%%%%%%%%%%%%%%%%%%%%%%%%%

\usepackage{amssymb}
\usepackage{amsmath}
\usepackage{amsfonts}
%\usepackage{algorithm}
%\usepackage{algorithmic}
%\usepackage{rotating} %--->se utiliza para rotar las tablas o el texto de una celda, implica \begin{sideways}...\end{sideways}
\usepackage{booktabs} %--->se utiliza para aumentar los espacios verticales y horizontales de las columnas de las tablas, por ejemplo, con el codigo: @{\vrule height 13pt depth 0pt width 0pt} en una de las columnas es suficiente para el alto de todas las columnas, asi como regular el ancho
%--->
%--->\parbox{4in}{texto} este es otro codigo importante para que las oraciones se coloquen en columnas en una tabla
\usepackage{hyperref}
%\usepackage[dvipdfm]{hyperref}%--- probar con dvipdfmx y backref=true
%\hypersetup{bookmarks,colorlinks=true,linkcolor=blue,citecolor=red} %---utilizar black para imprimir
%\hypersetup{bookmarks,colorlinks=true,linkcolor=black,citecolor=black} %---utilizar black para imprimir
%\usepackage[hypertex]{hyperref} 

%\theoremstyle{definition}
%\theoremstyle{theorem}
%\renewcommand{\refname}{Bibliograf\'{\i}a}
%\renewcommand{\contentsname}{\'{I}ndice}
%\renewcommand{\tablename}{Tabla}
%\AtBeginDocument{\renewcommand\tablename{Tabla}}
%\renewcommand{\figurename}{Figura}
%\renewcommand{\refname}{Bibliograf\'{\i}a}
%\renewcommand{\bibname}{Bibliograf\'{\i}a}

%\renewcommand{\contentsname}{\textsc{\'Indice general}} %-----Se colocan dentro de \begin{document}
%\renewcommand{\listfigurename}{\textsc{\'Indice de figuras}} %-----Se colocan dentro de \begin{document}
%\makeatletter\renewcommand{\ALG@name}{Algoritmo} %-----Se colocan dentro de \begin{document}

%%%---la revision del idioma se le puede cambiar en \Project\Propierties\es
%\NewDocumentCommand{\nombre}{}{codigo con #1, #2, ...}
%\newcommand{\dr}{ esdedf\textbf{bvfdbgfgb} } %---este es un ejemplo de comando utilizado para simplificar
%\newtheorem{identificador}[contador]{nombre del teorema}[section]
%\newtheorem{teor}{Teorema}[section] %--->este es un ejemplo
%\theoremstyle{definition}

%\addtolength{\textwidth}{0cm}
%\addtolength{\textheight}{0cm}
%\oddsidemargin 1.0cm
%\setlength{\textwidth}{15.59cm}
%\topmargin -1cm
%\setlength{\textheight}{24.94cm}

\frenchspacing
%\pagestyle{headings}
\pagestyle{myheadings}
\markright{\textsc{Curriculum Vitae 2023}\hspace*{8.5cm}{O. Tito-Corrioso}}

%--------------------------------------------------------
%\thispagestyle{plain}
%\begin{center}
%
%\includegraphics[width=3cm,height=3cm]{sello1.eps}
%\\\vspace*{0.2cm}
%{\Large \bf \textsc{Asignatura Optativa III (Criptograf\'ia)}}\\
%\vspace*{0.2cm}
%{\Large \bf \textsc{Trabajo Extra Clase}}\\
%\vspace*{0.3cm}
%{\bf Osmani Tito Corrioso}\\
%
%\vspace*{0.3cm}
%Departamento de Matem\'atica\\
%
%Facultad de Ciencias Naturales y Exactas\\
%
%Universidad de Oriente\\
%\vspace*{0.3cm}
%\today
%
%\end{center}
%----------------------------------------------------
%\title{\bf{Asignatura Optativa I\\
%Teor\'ia de N\'umeros y Criptograf\'ia}}
%\author{{\bf Osmani Tito Corrioso}\\[0.3cm]
%{\small{\bf Lic. en Matem\'atica, 3er A\~no, Universidad de Oriente}}}
%\date{{\bf Junio de 2016}}
%
%\maketitle

%\setcounter{secnumdepth}{-1}
%\setcounter{secnumdepth}{6}
%\pagenumbering{roman}
%\pagenumbering{arabic}

%\DeclareGraphicsExtensions{.jpg,.pdf,.png,.eps}
%\includegraphics[width=7cm,height=6cm]{figura1.eps}

%\bibliographystyle{ESTILO}%los estilos pueden ser: plain, apalike, alpha, abbrv, unsrt
%\bibliography{basededatos1[,basededatos2,...]}
%%--para las citas se tiene:  \nocite{Llave},  \nocite{*}

%\begin{thebibliography}{99}
%\addcontentsline{toc}{section}{Referencia}
%\bibitem[DK07]{} Delfs, Hans and Knebl, Helmut (20017). \textit{Introduction to Cryptografy: Principles and Applications}. Information Security and Cryptography Texts and Monographs. Second Edition. Springer. Library of Congress Control Number: 2007921676. ISSN 1619-7100. ISBN-13 978-3-540-49243-6.
%\end{thebibliography}


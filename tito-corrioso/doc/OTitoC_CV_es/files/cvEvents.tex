
\section{{Eventos}}

\begin{tabular}{p{3cm}p{.75\linewidth}}
\hfill\bfseries\scshape 2022 & $\bullet$ VI Seminario Cient\'ifico Nacional de Criptograf\'ia. \textit{``Mejorando la b\'usqueda de soluciones de sistemas MRHS mediante el Algoritmo Gen\'etico''}. \textbf{Osmani Tito-Corrioso}, Mijail Borges-Quintana, Miguel A. Borges-Trenard. La Habana, Cuba.
\end{tabular}

\begin{tabular}{p{3cm}p{.75\linewidth}}
\hfill\bfseries\scshape 2021 & $\bullet$ VII Taller Internacional: La Matem\'atica: su ense\~nanza, su aprendizaje y su did\'actica, MatGu' 2021. \textit{``Evaluaci\'on competitiva en el proceso de ense\~nanza - aprendizaje de la Matem\'atica''}. Beronica Aguilar-Le\'on, \textbf{Osmani Tito-Corrioso}. Guant\'anamo, Cuba.\\
& $\bullet$ XVII Congreso Internacional de Matem\'atica y Computaci\'on COMPUMAT 2021. \textit{``Sobre la soluci\'on de los sistemas MRHS mediante el Algoritmo Gen\'etico''}. \textbf{Osmani Tito-Corrioso}, Miguel A. Borges-Trenard, Mijail Borges-Quintana. La Habana, Cuba.
\end{tabular}

\begin{tabular}{p{3cm}p{.75\linewidth}}
& $\bullet$ II Convenci\'on Internacional Ciencia y Conciencia de la Universidad de Oriente. \textit{``Medio siglo de la primera graduaci\'on de Matem\'atica en la Universidad de Oriente: Aproximaci\'on hist\'orica''}. Enrique de Jes\'us Estrada-Pato, \textbf{Osmani Tito-Corrioso}, \textit{et. al}. Santiago de Cuba, Cuba.
\end{tabular}

\begin{tabular}{p{3cm}p{.75\linewidth}}
\hfill\bfseries\scshape 2020 & $\bullet$ V Seminario Cient\'ifico Nacional de Criptograf\'ia. \textit{``An\'alisis y dise\~no de variantes del criptoan\'alisis a cifrados en bloques mediante el Algoritmo Gen\'etico''}. \textbf{Osmani Tito-Corrioso}, Miguel A. Borges-Trenard, Mijail Borges-Quintana. La Habana, Cuba.
\end{tabular}

\begin{tabular}{p{3cm}p{.75\linewidth}}
\hfill\bfseries\scshape 2019 & $\bullet$ III Simposio Internacional Ciencia e Innovaci\'on Tecnol\'ogica 2019, Red de Investigadores de la Ciencia y la T\'ecnica, Cap\'itulo Guant\'anamo. ``Sobre la partici\'on del espacio de las claves''. \textbf{Osmani Tito-Corrioso}, Miguel A. Borges-Trenard, Mijail Borges-Quintana. La Habana, Cuba.
\end{tabular}

\begin{tabular}{p{3cm}p{.75\linewidth}}
& $\bullet$ XVI Congreso Internacional de Matem\'atica y Computaci\'on COMPUMAT 2019. \textit{``Sobre la pertenencia de las claves a clases de equivalencia en $G_{K}$''}. \textbf{Osmani Tito-Corrioso}, Miguel A. Borges-Trenard, Mijail Borges-Quintana. La Habana, Cuba.
\end{tabular}

\begin{tabular}{p{3cm}p{.75\linewidth}}
\hfill\bfseries\scshape 2018 & $\bullet$ IV Seminario Cient\'ifico Nacional de Criptograf\'ia. \textit{``Ataques a cifrados en bloques mediante b\'usquedas en grupos cocientes de las claves''}. \textbf{Osmani Tito-Corrioso}, Miguel A. Borges-Trenard, Mijail Borges-Quintana. La Habana, Cuba.
\end{tabular}

\begin{tabular}{p{3cm}p{.75\linewidth}}
& $\bullet$ I$^{ra}$ Jornada Cient\'ifica ANTEX ``SALUD CUBA - ANGOLA 2018''. \textit{``Modelos de la din\'amica de dispersi\'on de la Malaria mediante Ecuaciones Diferenciales''}. Adolfo Fern\'andez G., Beronica Aguilar L., Sandy S\'anchez, Iv\'an Ruiz, \textbf{Osmani Tito-Corrioso}. Huambo, Angola.
\end{tabular}

\begin{tabular}{p{3cm}p{.75\linewidth}}
& $\bullet$ F\'orum Cient\'ifico Estudiantil. \textit{``Influencia de ciertos par\'ametros en un modelo de la dispersi\'on del Dengue''}. Beronica Aguilar Le\'on, \textbf{Osmani Tito-Corrioso}, Adolfo A. Fern\'andez Garc\'ia. FCNE, Universidad de Oriente. \textbf{\textit{Premio Relevante}}.
\end{tabular}

\begin{tabular}{p{3cm}p{.75\linewidth}}
& $\bullet$ Primer Taller de Historia y Patrimonio de la Universidad de Oriente. \textit{``A 50 a\~nos de la carrera de Matem\'atica en la Universidad de Oriente: apuntes para una historia''}. Enrique de J. Estrada-Pato, \textbf{Osmani Tito-Corrioso}, Francisco R. Mart\'inez-S\'anchez.
\end{tabular}

\begin{tabular}{p{3cm}p{.75\linewidth}}
& $\bullet$ F\'orum CIENES UO 2018. \textit{``Influencia de ciertos par\'ametros en un modelo de la dispersi\'on del Dengue''}. Beronica Aguilar Le\'on, \textbf{Osmani Tito-Corrioso}, Adolfo A. Fern\'andez Garc\'ia. Universidad de Oriente.
\end{tabular}

\begin{tabular}{p{3cm}p{.75\linewidth}}
\hfill\bfseries\scshape 2017 & $\bullet$ F\'orum Cient\'ifico Estudiantil. \textit{``Apuntes sobre la historia de las Ciencias Matem\'aticas en Santiago de Cuba''}. Enrique de J. Estrada Pato, \textbf{Osmani Tito-Corrioso}. FCNE, Universidad de Oriente. \textbf{\textit{Premio Relevante}}.\\
& $\bullet$ F\'orum de Historia de la FEU de la UO. \textit{``Apuntes sobre la historia de las Ciencias Matem\'aticas en Santiago de Cuba''}. Enrique de J. Estrada Pato, \textbf{Osmani Tito-Corrioso}. Universidad de Oriente. \textbf{\textit{Premio Relevante}}.
%\hfill\bfseries\scshape  & \\
%\hfill\bfseries\scshape  & 
\end{tabular}

\begin{tabular}{p{3cm}p{.75\linewidth}}
\hfill\bfseries\scshape 2016 & $\bullet$ IV Taller Nacional \'Alvaro Reynoso. \textit{``Monitorizaci\'on remota y productividad de la colmena''}. \textbf{Osmani Tito-Corrioso}, Rayner Cadrelo Rodr\'iguez. Universidad de Oriente.\\
& $\bullet$ F\'orum de Historia de la FEU de la UO. \textit{``Historia de la Matem\'atica en Cuba''}. Enrique de J. Estrada Pato, \textbf{Osmani Tito-Corrioso}. Universidad de Oriente. \textbf{\textit{Premio Destacado}}.
%\hfill\bfseries\scshape  & \\
%\hfill\bfseries\scshape  & 
\end{tabular}

\begin{tabular}{p{3cm}p{.75\linewidth}}
\hfill\bfseries\scshape 2015 & $\bullet$ Primer Taller de Historia de la Universidad de Oriente. \textit{``Historia de la Matem\'atica en la UO''}. Enrique de J. Estrada Pato, \textbf{Osmani Tito-Corrioso}. Universidad de Oriente.
\end{tabular}

\begin{tabular}{p{3cm}p{.75\linewidth}}
& $\bullet$ XXII Jornada Cient\'ifica Estudiantil de la Universidad de Oriente CIENES 2015. \textit{``Determinaci\'on de la fuerza realizada durante el proceso din\'amico de fresado con altas velocidades''}. Alfredo Dur\'an Parada, \textbf{Osmani Tito-Corrioso}. \textbf{\textit{Premio Menci\'on}}.
\end{tabular}

\begin{tabular}{p{3cm}p{.75\linewidth}}
& $\bullet$ F\'orum Cient\'ifico Estudiantil de la Facultad de Matem\'atica-Computaci\'on. \textit{``Determinaci\'on de la fuerza realizada durante el proceso din\'amico de fresado con altas velocidades''}. Alfredo Dur\'an Parada, \textbf{Osmani Tito-Corrioso}. Universidad de Oriente. \textbf{\textit{Premio Relevante}}.
%\hfill\bfseries\scshape  & \\
%\hfill\bfseries\scshape  & 
\end{tabular}

%\begin{tabular}{p{3cm}p{.75\linewidth}}
%\hfill\bfseries\scshape  & \\
%\hfill\bfseries\scshape  & \\
%\hfill\bfseries\scshape  & 
%\end{tabular}

%\begin{tabular}{p{3cm}p{.75\linewidth}}
%\hfill\bfseries\scshape  & \\
%\hfill\bfseries\scshape  & \\
%\hfill\bfseries\scshape  & 
%\end{tabular}